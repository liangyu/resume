%%%%%%%%%%%%%%%%%%%%%%%%%%%%%%%%%%%%%%%%%
% Important notes:
% This template needs to be compiled with XeLaTeX and the bibliography, if used,
% needs to be compiled with biber rather than bibtex.
%\documentclass[]{ly-cv} 
%%%%%%%%%%%%%%%%%%%%%%%%%%%%%%%%%%%%%%%%%
\documentclass[]{ly-cv} % Add 'print' as an option into the square bracket to remove colors from this template for printing
\begin{document}
\header{liang}{yu}{A Web, Mobile and Backend Developer} % Your name and current job title/field
%----------------------------------------------------------------------------------------
%	SIDEBAR SECTION
%----------------------------------------------------------------------------------------
\begin{aside} % In the aside, each new line forces a line break
\section{contact}
Malden, MA
(617) 901 0618
\href{mailto:yuliang.name@gmail.com}{yuliang.name@gmail.com}
\href{http://www.linkedin.com/in/liangyu2012neu}{LinkedIn}
\section{programming languages}
Java python c cpp 
shell \LaTeX
\section{operating system}
Mac OS X Ubuntu Windows
\section{web development}
Javascript Ajax REST
CSS3 HTML5 Dojo GWT
\section{mobile development}
Android Windows 8
\section{development tools}
Eclipse SublimeText VIM
GIT Hadoop MapReduce
Xen Fiddler GDB Jenkins 
MySQL Firebug Windows Azure
\end{aside}

%----------------------------------------------------------------------------------------
%	EDUCATION SECTION
%----------------------------------------------------------------------------------------

\section{education}

\begin{entrylist}
%------------------------------------------------
\entry
{2010-2012}
{Masters of Science} 
%{Masters {\normalfont of Science}}
{Northeastern Unviersity, Boston, United States}
{Computer Science \hspace{4mm} \footnotesize{GPA 3.54/4.0}}
%{\emph{Computer Science} \\ \footnotesize{GPA 3.5/4.0}}
%------------------------------------------------
\entry
{2005-2009}
{Bachelor of Science}
{Nanchang University, Nanchang, China}
{Information and Computational Science}
%------------------------------------------------
\end{entrylist}
%----------------------------------------------------------------------------------------
%	WORK EXPERIENCE SECTION
%----------------------------------------------------------------------------------------
\section{experience}
\begin{entrylist}
%------------------------------------------------
\entry
{2013 - Now}
{Integrated Computer Solutions}
{Bedford, Massachusetts}
{\emph{Mobile Software Developer}\\
HTTPS Server for Topics
\begin{itemize}
  \item Build a sync model to hand data-sync between android device and server.
  \item Used SQLite Database and sharedprefrences file to save data, and used SQL to selected data for three views. 
  \item Used adapters to automatically bind data to the view.
  \item Implemented AsyncTask subclasses to handle the send/receive data between android device and server.
  \item used Agile method to work with other developers.
  \item Used OO design in writing asynctask subclasses.
  \item used fragment and navigation drawer to build the UI.
\end{itemize}
Android App for Topics
\begin{itemize}
  \item Modified existed Code to add advertisements in the Game by leveraging the Microsfot Advertsing SDK.
\end{itemize}
Drupal for Web Team
\begin{itemize}
  \item Modified existed Code to add advertisements in the Game by leveraging the Microsfot Advertsing SDK.
\end{itemize}
Inkarus Game
\begin{itemize}
  \item Modified existed Code to add advertisements in the Game by leveraging the Microsfot Advertsing SDK.
  \item Used the WinJS library and HTML 5 to added two dialogs to submit results to the Windows Azure Serve and retrieve the others' result from the serve.
  \item Fixed minor bugs in the legacy codes.
\end{itemize}
Questia
\begin{itemize}
  \item Leveraged the Model-View-Viewmodel, Service Locator, Observer patterns in the development.
  \item Used XAML to build views and binded them to the datamodels, which are written in C\#.
  \item Used the Fiddler to analyze the data transferred between the client and the Questia Server.
\end{itemize}}
%------------------------------------------------
\entry
{2012}
{Acme Packet}
{Bedford, Massachusetts}
{\emph{Co-op Enginner Intern}
\begin{itemize}
  \item Helped with ASC automated testing framework by writing twenty pyton tests on SIP protocol.
	\item Identified bugs in four existing web services application, which were written separately in Java, GWT/GXT, JavaScript and Flex
  \item Developed the Call-Transfer web services application with JavaScript and Dojo API.
	\item Implemented performance tests in Python scripts to measure product performance.
  \item Enhanced the quality of an existing iOS mobile application by fixing two bugs and improved its UI and workflow by implementing good interface design principles.
\end{itemize}}
%------------------------------------------------

\end{entrylist}

%----------------------------------------------------------------------------------------
%	AWARDS SECTION
%----------------------------------------------------------------------------------------

\section{relevant coursework}

\begin{entrylist}
%------------------------------------------------Analyzing Company Stock Prices Via Twitter Posts
\entry
{graduate}
{Big Data, Cloud Computing}
{Northeastern University}
{courses: \emph{Parallel Computing in MapReduce, Introduction to Database}\vspace*{4pt}}
%------------------------------------------------
\end{entrylist}
\begin{entrylist}
%------------------------------------------------
\entry
{}
{Computer System}
{Northeastern University}
{courses: \emph{Computer System, High Performance Computing}\vspace*{4pt}}
%------------------------------------------------
\end{entrylist}
\begin{entrylist}
%------------------------------------------------
\entry
{}
{Software Design}
{Northeastern University}
{courses: \emph{Program Design Paradigms, Managing Software Development}\vspace*{4pt}}
%------------------------------------------------
\end{entrylist}
\begin{entrylist}
%------------------------------------------------
\entry
{}
{Algorithms}
{Northeastern University}
{courses: \emph{Algorithms, Advanced Algorithms}}
%------------------------------------------------
\end{entrylist}
\end{document}